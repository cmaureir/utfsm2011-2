The strip-packing problem takes a relevant role
in many applications of business and industry,
like wood, glass, metal, paper, etc,
because working with materials will need in
some moments a properly use of all the possible resources,
for example, cutting pieces of steel will generate material remains,
which in other words are a considerable cost to the business,
so is necessary to optimize process like this.

Like most of the optimization problems,
depending of the applied area,
some of the objectives and constraints will change,
but all follow the basic idea to optimize the
sub-sections distribution  in a given space.

In simple words, the problem consist in determine if a
set of items can be grouped in a section with a fixed size,
to a better understanding, we can think in a big rectangle
space which need to be filled with several smaller rectangles.
Furthermore, the main objective of this problem needs to
satisfy some hard-constraints, which are not generate
pieces overlapping, and follow the dimensions
of the given section.

In the following section,
we present a brief revision of some recent related works,
to be able to generate a simple scenario to generate
new solution proposals to the strip-packing problem.
