% TO DO
% ambiente donde se corrieron los tests
% tablas con resultados
\subsection{Hardware configuration}

The hardware details of the used computer are described in \ref{tab:table1}.

\begin{table}
    \centering
    \begin{tabular}{|l|r|}
        \hline
        \textbf{Processor} & Phenom II X4 965 Black AM3 3.4Ghz\\\hline
        \textbf{RAM} & 8GB 1333MHz\\\hline
        \textbf{OS} & Arch Linux, Kernel 3.2.11 \\\hline
    \end{tabular}
    \label{tab:table1}
    \caption{Hardware configuration}
\end{table}

\subsection{Experiments instances}

The used parameters are:
\begin{itemize}
    \item Selection parameter ($b = 10$)
    \item Solutions number    ($popsize = b*5$)
    \item Maximum iteration   ($max\_iter = 100$)
\end{itemize}

\begin{table}[h!t]
    \centering
    \small
    \begin{tabular}{|c|c|c|c|c|}
        \hline
        \textbf{Instance} & \textbf{Elements} & \textbf{Time} & \textbf{Height} & \textbf{Height (*)} \\ \hline
        c1-p1 (HT01)      & 17                & 0.29s          & 20              & 20                  \\ \hline
        c1-p2 (HT02)      & 18                & 4.23s          & 21              & 20                  \\ \hline
        c1-p3 (HT03)      & 17                & 0.21s          & 20              & 20                  \\ \hline
        c2-p1 (HT04)      & 26                & 3.97s          & 16              & 15                  \\ \hline
        c2-p2 (HT05)      & 26                & 4.17s          & 16              & 15                  \\ \hline
        c2-p3 (HT06)      & 26                & 5.87s          & 16              & 15                  \\ \hline
        c3-p1 (HT07)      & 29                & 28.73s         & 32              & 30                  \\ \hline
        c3-p2 (HT08)      & 30                & 33.06s         & 33              & 30                  \\ \hline
        c3-p3 (HT09)      & 29                & 40.08s         & 32              & 30                  \\ \hline
        c4-p1             & 49                & 11m 3.01s      & 63              & 60                  \\ \hline
        c4-p2             & 49                & 8m 57.31s      & 63              & 60                  \\ \hline
        c4-p3             & 49                & 4m 0.39s       & 63              & 60                  \\ \hline
        c5-p1             & 73                & 10m 20.90s     & 95              & 90                  \\ \hline
        c5-p2             & 73                & 16m 3.84s      & 95              & 90                  \\ \hline
        c5-p3             & 73                & 11m 13.9s      & 95              & 90                  \\ \hline
        c6-p1             & 98                & 40m 15.9s      & 127             & 120                 \\ \hline
        c6-p2             & 98                & 79m 51.48s     & 127             & 120                 \\ \hline
        c6-p3             & 98                & 43m 34.53s     & 127             & 120                 \\ \hline
        c7-p1 (+)         & 196               & $>$160m        & 172             & 240                 \\ \hline
        c7-p2 (+)         & 197               & $>$160m        & 171             & 240                 \\ \hline
        c7-p3 (+)         & 196               & $>$160m        & 172             & 240                 \\ \hline
    \end{tabular}
    \label{tab:results}
    \caption{Hopper and Turton benchmark data. (*) means the optimal height. (+) strange behaviour}
\end{table}

The present implementations shows a good behaviour with the first nine
instances, in terms of execution time and obtained height.

Consiring the classes 4,5 and 6,
the execution time grows significantly,
but the obtained solution height follows the same characteristics
of the previous one, a difference between 3 and 5.

The implemented function which perform the placement heuristics (BLF),
takes around the $90\%$ of the execution time, and considering the amount of
elements inf the last three instances, the algorithm was stopped in the 5th iteration,
due the lack of performance.

Another critical issue is that the most difficult instances
produce a strange behaviour in the implemented solution,
returning no valid heights. This point is a very important section
which needs to be fixed in the future work.

