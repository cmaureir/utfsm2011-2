Today, there are numerous papers around the
strip-packing problem, which implies that is not possible
to consider all of them, including all the meta-heuristics,
and heuristics methods.

A complete compendium of the most representative
solution to the present problem can be found in the work
presented in Hooper (2000)\cite{hooper},
additionally the work presented by Riff et al. (2004)\cite{riff},
show a brief and simple analysis of some important
papers between both published dates.

This section aims to collect some actual good approaches
to solve the strip-packing problem.


\subsection{Based on best-fit heuristic}

In the work presented by Burke et al. (2010)~\cite{burke}
a new methodology was presented which modify the
way to determine the penalization of each item position.

The base methodology is find the difficult items (high penalty),
and use them first, leaving the easiest items (low penalty)
positioning, to the end of the algorithm.

The algorithm search the item positions which break bound
constraints and assign it a penalization based on the existing
penalization plus the size of the height (considering filling the space
from above).

% "A squeaky wheel optimisation methodology for two dimensional strip packing"
%
% Problema:
%     - Wäscher et al 2007
%     - 2D orthogonal open dimension problem.
%     - Not using piece rotations
%     - Guillotine cuts are not required
%     - referenciado como el subtipo "Oriented, Free cutting" (OF)    Lodi et. al (1999) Bortfeldt (2006).
%
%     -Conceptos del best-fit
%         -new piece in the lowest available slot, structure update.
% Objetivo
%     -simple y efectiva methodology, basada en squeakly wheel optimisation,
%         identificar dificiles construyendo soluciones,

\subsection{Non-based on bottom-left heuristic}




\subsection{Exact methods}


One of the exact methods to solve the strip-packing problem,
is the branch and bound approach,
which only consider solutions using all the necessary items
to fill the entire space.
The idea behind this approach, is solve the situations when
is not possible to add a new item in the space,
without generate an empty space (hole),
so the algorithm start to cut branches.

In the presented work by Alvarez-Valdes et al. (2009)\cite{alvarez},
the focus is over add new conditions to the strips,
to be able to generate better solutions,
also reduces the problem to different knapsack problems using the slides,
before apply the lower bounds.
Almost all the main ideas of the implementations,
are based on the work presented by Martello et al.(2003)\cite{martello},
chiefly the idea of use computed lower bound,
relaxing some constraints of the items surface,
dividing each item in slides to search solutions
more easily, but also considering that at the end of the algorithm,
is essential to join all the slides of the different items.








% "A branch and bound algorithm for the strip packing problem"
% 
% - Desarrollan lower bounds basados in formuationes de relajaciones enteras
% - Basan la mayoria del algoritmo (structure of the search tree) de Martello et al 2003
% - Proponen nuevas lower bounds
% - Primer se aplica la reducción (utilizan GRASP para ver si las piezas caben en un rectangulo)
% - Despues terminan las tiras y piezas para resolver un problema de la mochila (knapsack)
