This paper has presented a scatter-search meta-heuristic,
which contains different methodologies to improve
the reference set used in each algorithm step,
as the \emph{iterative shift}, a simple \emph{swap}
and a PMX,
obtaining good solutions in a few iterations.

The use of BLF as placement heuristics,
give us the possibility to avoid big holes in the generated
solutions, obtaining results near to the optimal.

There is also a brief literature review,
with some important actual publications
that provide efficient solutions to the
strip-packing problem.

About the future work,
there are several details in the scatter-search,
which needs to be careful, overall
controlling the relation between exploration
and exploitation, because the presented solution
has some bad behavior with large instances,
due two main issues:

\begin{itemize}
    \item Movements with a high computational cost,
        which considering large instances, produce
        solutions in a very slow way.
    \item Due all the effort in each movement,
        the solution has stagnation problems,
        which can be solved, using for example,
        parameters control.
\end{itemize}

Finally,
the next step has in consideration,
to improve the proposal to obtain
good solutions in less time that the actual
implementation.
